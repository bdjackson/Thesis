%% \chapter[htoc-titlei][hhead-titlei]{htitlei}
%% -----------------------------------------------------------------------------
\chapter[General introduction][General introduction]{General introduction}
\label{ch:intro}

This thesis documents the search for direct scalar top pair production, with
the decay of each stop via an $R$-parity-violating (RPV) interaction to a
charged lepton (electron or muon) and a $b$-quark, as shown in
Figure~\ref{fig:blstop_diagram}.
The analysis uses 20.3~\ifb of $\sqrt{s}=8~\TeV$ proton-proton collision data
collected with he ATLAS detector at the Large Hadron Collider (LHC).

\begin{figure}[ht]
  \centering
  \includegraphics[width=0.60\textwidth]{figs/blstop/b_minus_l_stop_stop.pdf}
  \caption{Simplified model of pair production of stop quarks, with decay to a
    charged lepton and $b$-quark.
  }
  \label{fig:blstop_diagram}
\end{figure}


The experimental signature is two oppositely charged leptons and two identified
$b$-jets.
The analysis considers $eebb$, $e \mu bb$, and $\mu \mu bb$ final states.
Final states with $\tau$ leptons are not considered for this search.
The distinguishing features are two pairs, each of a lepton and a $b$-jet, with
a resonance in the invariant mass distribution of each pair.
In contrast to $R$-parity conserving searches, there is no significant missing
transverse momentum.

Previous searches for lepto-quarks at ATLAS~\cite{ATLAS:2013oea,
ATLAS:2012aq, Aad:2011ch, Aad:2011uv} and
CMS~\cite{Khachatryan:2014ura, CMS:2014qpa, Chatrchyan:2012sv,
Chatrchyan:2012vza} have considered pair production of first, second,
and third generation lepto-quarks, but have not examined the signature
of a resonance in the invariant mass of an electron and a $b$-jet or a
muon and a $b$-jet.  The results of these searches have already been
interpreted to set limits on the stop mass and its decay
branching fractions in the $B-L$
model~\cite{Marshall:2014cwa, Marshall:2014kea}.

Chapter~\ref{ch:theory} gives a description of the Standard Model (SM) of
particle physics and Supersymmetry, a popular extension to the SM.
The concept of $R$-parity, and the $B-L$ extension to the SM is also discussed
in this chapter.
Emphasis is placed on the phenomenology of this model.

The LHC and ATLAS detector, along with the triggering system and event
reconstruction is also described in Chapter~\ref{ch:lhc}.
Chapter~\ref{ch:trt} describes the Transition Radiation Tracker sub-detector
in more detail.
{\color{red} TODO decide if this chapter actually belongs in the final version}
The Monte Carlo (MC) event generator tools used for estimating the detector
response and efficiency to reconstruct the signal process, estimate systematic
uncertainties, and to predict the backgrounds from SM processes, are described
in Chapter~\ref{ch:mc}.

Chapter~\ref{ch:bl_stop} describes the stop search.
This chapter reviews the search strategy, event selection, and interpretation
of the results.
The chapter concludes with a brief description of proposed improvements to the
analysis.
{\color{red} TODO decide if this section actually fits in the final version}
