%% \chapter[htoc-titlei][hhead-titlei]{htitlei}
%% -----------------------------------------------------------------------------
\chapter[Theoretical overview][Theory]{Theoretical overview}
\label{ch:theory}

{\color{red} I'll need to go back and re-read through some books/papers before
  writing this chapter!}

{\color{red} Evelyn: 
keep this brief by referencing as much as possible.
Certainly contrast assumption of RPC vs assumption of RPV for experimental
searches.
Both are possibilities.
Stop as LSP does give unique (crazy?) signature not well
tested by previous searches (include LQ exclusion plot).
Could have different
decay rates to e, mu, tau.
}

%% ------------------------------------------------------------------------------
\section{Standard Model}

%% ------------------------------------------------------------------------------
\section{Supersymmetry}

%% - - - - - - - - - - - - - - - - - - - - - - - - - - - - - - - - - - - - - - -
\subsection{R-Parity}

The extension of the Standard Model of particle physics with
supersymmetry (SUSY)~\cite{Miyazawa:1966,Ramond:1971gb,Golfand:1971iw,
Neveu:1971rx,Neveu:1971iv,Gervais:1971ji,Volkov:1973ix,Wess:1973kz,Wess:1974tw}
immediately leads to processes that violate both baryon number ($B$) and
lepton number ($L$), leading to rapid proton decay and
lepton-number-violating processes, such as unseen decays of
$\mu \rightarrow e\gamma$, in conflict with experimental bounds.
A conventional assumption to prevent these processes is to impose
conservation of $R$-parity~\cite{Fayet:1976et,Fayet:1977yc,Farrar:1978xj,
Fayet:1979sa,Dimopoulos:1981zb},
defined as $R=(-1)^{3(B-L)+2s}$ where $s$ is the spin of the particle.
This has a value of $+1$ for Standard Model particles and $-1$ for
SUSY particles.
In this case SUSY particles are produced in pairs, and the
lightest supersymmetric particle (LSP) is stable.
Further, this stable LSP cannot carry electric charge or color charge without
coming into conflict with astrophysical data.
At the LHC, the conventional experimental signature for SUSY particles
includes significant missing transverse momentum due to the non-interaction of
the LSP with the detector.

%% -----------------------------------------------------------------------------
\section{B-L extension}
\label{sec:theory_bl_extension}


%% - - - - - - - - - - - - - - - - - - - - - - - - - - - - - - - - - - - - - - -
\subsection{Motivation}

{\color{red} Why this model? :-) Talk about things like:
\begin{itemize}
\item RPC is overkill to prevent proton decay
\item Links neutrino sector to susy properties
\end{itemize}
}

An alternative approach is to add a local symmetry $U(1)_{B-L}$ to the
$SU(3)_C \times SU(2)_L \times U(1)_Y$ Standard Model with right-handed
neutrinos.
The minimal supersymmetric extension then only needs a vacuum expectation value
for a right-handed sneutrino in order to spontaneously break the
$B-L$ symmetry~\cite{FileviezPerez:2008sx, Barger:2008wn, FileviezPerez:2009gr,
Everett:2009vy, Evans:1986ada, Lukas:1998yy, Braun:2005ux, Braun:2005nv,
Braun:2006ae, Ambroso:2009jd, Ambroso:2010pe, Ovrut:2012wg}.
This minimal $B-L$ model violates lepton number but not baryon number, and is
consistent with proton stability and the bounds on lepton number violation.
The LSP can now decay via $R$-parity-violating (RPV) processes, and may now
carry color and electric charge.

%% - - - - - - - - - - - - - - - - - - - - - - - - - - - - - - - - - - - - - - -
\subsection{Phenomenology}

{\color{red} Why this model? :-)
\begin{itemize}
\item $U(1)_{B-L}$ gauge boson
\item Natural choices for LSP? Why stop
\item relation between stop branching ratios and the neutrino hierarchy)!
\item Why look for $\tilde{t} \rightarrow b\ell$ instead of
  $\tilde{t} \rightarrow t\nu$? (handedness argument)
\end{itemize}
}

This leads to unique signatures~\cite{FileviezPerez:2012mj, Perez:2013kla,
Ovrut:2012wg, Ovrut:2014rba, Ovrut:2015uea} that are disallowed in conventional
models with $R$-parity conservation.
The case where the LSP is a scalar top (stop) is most interesting
since, in general, the large mass of the top quark acts to make the
lightest stop significantly lighter than the other squarks due to
renormalization group effects~\cite{Barbieri:1987fn,deCarlos:1993yy}.
The stop decays via an RPV interaction to a charged lepton (of any
flavor) and a $b$-quark.
The decay branching fractions to $e b$, $\mu b$, and $\tau b$ may be different
in a manner related to the neutrino mass
hierarchy~\cite{Marshall:2014cwa, Marshall:2014kea}.
