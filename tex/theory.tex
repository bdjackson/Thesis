%% \chapter[htoc-titlei][hhead-titlei]{htitlei}
%% -----------------------------------------------------------------------------
\chapter[Theoretical overview][Theory]{Theoretical overview}
\label{ch:theory}

{\color{red} I'll need to go back and re-read through some books/papers before
  writing this chapter!}

{\color{red} Evelyn: 
keep this brief by referencing as much as possible.
Certainly contrast assumption of RPC vs assumption of RPV for experimental
searches.
Both are possibilities.
Stop as LSP does give unique (crazy?) signature not well
tested by previous searches (include LQ exclusion plot).
Could have different
decay rates to e, mu, tau.
}

The universe, as we know it, comprises fundamental matter particles, and
four fundamental forces.
These forces include gravity, electromagnetism, and the strong and weak nuclear
forces.
Despite being the weakest of the four forces, gravity is certainly the most
familiar to everyday life, as it assures objects fall to the ground.
The Gravitational force is described by the general theory of relativity, and
to this day, a successful quantum theory of gravity has yet to be developed.
Since gravity is so weak, it does not significantly affect the physics at the
LHC energy scale, and can safely be ignored for the purpose of this thesis.
Of the remaining three forces,  electromagnetism and the weak force have been
shown to come from the same underlying interactions, called the electroweak
force.
The electroweak and strong forces are described by the Standard Model of
Particle Physics.

In this chapter, a brief overview of the theoretical background for this thesis
is presented.
The Standard Model and its shortcomings are discussed
in \cref{sec:sm,sec:sm_shortcomings}.
Supersymmetry, a popular extension to the Standard Model, is introduced in
Section~\ref{sec:susy}.
The chapter concludes with a presentation of the particular $B-L$ extension to
the SM, which is the focus of the search presented in this dissertation.
Section~\ref{sec:theory_bl_extension} includes presents a brief description of
the underlying theory, as well as some of the interesting phenomenology expected
in the scenario where the scalar top is the LSP.

%% ------------------------------------------------------------------------------
\FloatBarrier
\section{Standard Model}
\label{sec:sm}

In this section, the Standard Model (SM) of particle physics is described in
brief.
The SM is a very rich subject, and a more complete description can be found in
References~\cite{Agashe:2014kda,opac-b1131978,halzen1984quarks}.
The SM is a quantum field theory which encapsulates the current understanding
of the elementary particles, and their interactions, and has been developed,
and rigorously tested by experiments over the last fifty years.
In 2012, the final particle predicted by the SM, the ``Higgs boson`` was
discovered at CERN by the ATLAS and CMS collaborations, marking a great
achievement for the both the experimental collaborations and the theorists
who predicted the particle's existence.
The SM Lagrangian is a non-abelian gauge theory with symmetry group 
$\mathrm{SU}(3)_\mathrm{c} \times
\mathrm{SU}(2)_\mathrm{L} \times
U(1)_\mathrm{Y}$,
which describes the matter content of the universe, as well as the interactions
of the strong and electroweak forces.
The matter content in the SM is made up of fermions (spin \nicefrac{1}{2}),
called quarks and leptons.
Massless gauge bosons (spin 1) mediate the interactions of the electroweak and
strong forces.

\begin{figure}[ht]
  \centering{
    \includegraphics[width=\textwidth]{figs/theory/sm_particle_content.pdf}
    \caption{Fundamental particles described by the Standard Model.
      The masses of each particles are given in parentheses.
    }
    \label{fig:sm_particle_content}
  }
\end{figure}

The fermionic matter is arranged into three families, or generations, each
containing quarks and leptons.
These particles can be charged under each part of the SM symmetry group,
where the particle's charge determines how it interacts with each of the
corces.
Each generation contains two chiral left-handed quarks, arranged in a isospin
doublet, consisting of an up-type and a down-type quark.
There are also two chiral left-handed leptons, one with electric charge, and
a neutrino, which is electrically neutral.
As with the quarks, the two left-handed leptons are arranged into an isospin
doublet.
Finally, each fermionic family contains chiral right-handed counterparts to
the two quarks, and the charged lepton.
No right-handed neutrinos are included in the SM, as they would no react with
any known forces, as will be explained shortly.
The right-handed fermions are isospin singlets.
The three generations are essentially copies of one another, differing only in
the mass of the constituent particles, and the familiar world is made up
entirely of particles from the first generation.
The reason for exactly three generations, not more or less, remains a mystery.
A summary of the SM matter content, with the charges under the various
parts of the SM symmetry group is shown in Table~\ref{tab:sm_matter_content}.

\begin{table}
  \caption{Summary of the matter particles described by the SM, along with the
    associated quantum numbers.
    The quantum numbers include the spin, electric charge $Q$, the third
    component of weak isospin $T_3$, hypercharge $Y$, and the allowable color
    charges.
  }
  \label{tab:sm_matter_content}
  \begin{center}
    \begin{tabular}{ccccccccc}
      \toprule
      \multirow{2}{*}{Fermions} &
      \multicolumn{3}{c}{Generation} &
      \multirow{2}{*}{Spin} &
      \multirow{2}{*}{$Q$} &
      \multirow{2}{*}{$T_3$} &
      \multirow{2}{*}{$Y$} &
      \multirow{2}{*}{Color}
      \\[1ex]
      & 1 & 2 & 3
      \\
      \midrule
      \addlinespace[1ex]
      %%
      \multirow{5}{*}{Quarks} &
      \multirow{2}{*}{$\left(
        \begin{tabular}{c} u \\ d \end{tabular} \right)_{L}$ } & % (u d) quarks
      \multirow{2}{*}{$\left(
        \begin{tabular}{c} c \\ s \end{tabular} \right)_{L}$ } & % (c s) quarks
      \multirow{2}{*}{$\left(
        \begin{tabular}{c} t \\ b \end{tabular} \right)_{L}$ } & % (t b) quarks
      \multirow{2}{*}{$\frac{1}{2}$} & % Spin
      $+\frac{2}{3}$ & % Q
      $+\frac{1}{2}$ & % T3
      \multirow{2}{*}{$\frac{1}{3}$} & % Y
      \multirow{2}{*}{r, g, b} % color
      \\[1ex]
      %%
      & & & & &
      $-\frac{1}{3}$ &
      $-\frac{1}{2}$ &
      &
      \\
      %%
      \cmidrule{2-9}
      %%
      &
      $u_R$ &
      $c_R$ &
      $t_R$ &
      $\frac{1}{2}$ &
      $+\frac{1}{2}$ &
      0 &
      $+\frac{2}{3}$ &
      r, g, b
      \\[1ex]
      %%
      &
      $d_R$ &
      $s_R$ &
      $b_R$ &
      $\frac{1}{2}$ &
      $-\frac{1}{2}$ &
      0 &
      $-\frac{1}{3}$ &
      r, g, b
      \\
      %%
      \midrule
      %%
      \multirow{3}{*}{Leptons} &
      \multirow{2}{*}{$\left(
          \begin{tabular}{c} $\nu_{e}$ \\ e \end{tabular}
        \right)_{L}$ } &
      \multirow{2}{*}{$\left(
          \begin{tabular}{c} $\nu_{\mu}$ \\ $\mu$ \end{tabular}
        \right)_{L}$ } &
      \multirow{2}{*}{$\left(
          \begin{tabular}{c} $\nu_{\tau}$ \\ $\tau$ \end{tabular}
        \right)_{L}$ } &
      \multirow{2}{*}{$\frac{1}{2}$} &
      0 &
      $+\frac{1}{2}$ &
      \multirow{2}{*}{-$\frac{1}{2}$} &
      \multirow{2}{*}{-}
      \\[1ex]
      %%
      & & & & &
      $-1$ &
      $-\frac{1}{2}$ &
      &
      \\
      \cmidrule{2-9}
      %%
      &
      $e_{R}$ &
      $\mu_{R}$ &
      $\tau_{R}$ &
      $\frac{1}{2}$ &
      $-1$ &
      0 &
      $-1$ &
      -
      \\
      \bottomrule
    \end{tabular}
  \end{center}
  
\end{table}

In addition to a charge, each piece of the SM symmetry group is associated with
a massless gauge boson, which mediates the interactions between particles.
The electroweak sector has the symmetry group
$\mathrm{SU}(2)_\mathrm{L} \times U(1)_\mathrm{Y}$, and describes the
interactions with fields with isospin or hypercharge ($Y$) in a theory called
quantum electrodynamics (QED).
The electroweak sector is a chiral theory, meaning right- and left-handed
particles transform differently under the gauge group.
Left-handed fermions have a value of the third component of isospin equal to
$\pm \nicefrac{1}{2}$, while right-handed particles have no isospin, and do
not interact with the weak force.
The gauge bosons associated with the electroweak sector are three $W^{i}$
bosons, arranged in a weak isospin triplet, and a $B^0$ boson, which is a weak
isospin singlet.

Since chiral fields translate differently, depending on their chiral handedness,
there can be no mixing between the right- and left-handed states, which would
break gauge invariance.
Unfortunately, a mass term, allows for exactly this mixing, implying the
fermions, as well as the gauge bosons must be massless.
On the other hand, the weak force is observed to be short ranged, implying
massive gauge bosons, and fermions do indeed have measurable masses.
Adding masses to a chiral quantum field theory is a difficult endeavor, and
is achieved through a process called ``spontaneous symmetry breaking,'' which
is explained in Section~\ref{sec:higgs}.

The $SU(3)_c$ part of the SM Lagrangian corresponds to the strong force,
with a corresponding ``color'' charge.
Quarks are colored objects, having red, green, or blue charge, while leptons
have no color charge, and thus do not interact directly with the strong
force.
The eight massless gauge bosons associated with the $SU(3)_c$ symmetry group
are called gluons, and are themselves colored objects.
This leads to self interaction, and some interesting phenomenology!

Because the gluon are allowed to interact with one another, QED and QCD have
some very important differences.
Perhaps, the most interesting, is related to the concept of ``screening''.
In QED, as one moves further from a charged paricle, the charge tends to look
smaller as a result of the polarization of the vacuum around the charge, where
electron pairs pop out of the vacuum.
The further from the particle one is, the more electron pairs are visible,
obscuring the initial charge more.
In QCD, this same screening effect occurs, with quark-anti-quark pairs being
pair-produced from the vacuum, however gluon pairs may also be produced out of
the vacuum.
These gluon pairs produce the opposite screening effect, known as
``anti-screening,'' increasing the effective charge observed.
Rather than charges being screened, as in QED, there is a net anti-screening
effect for objects with color charge.

As two colored particles get very close together, the anti-screening is very
small, and they can be treated as free particles, in an effect known as
``asymptotic freedom.''
Alternatively, as two charged particles move apart, the polarization of the
vacuum between them results in an increasing energy buildup, until it is
energetically favorable to pair produce a quark-anti-quark pair from the vacuum.
For this reason, no free quarks or gluons have been observed, rather any
objects with color charge will tend to into bound states called ``hadrons,''
which has no net color.
Hadrons can either be a bound state of a quark-anti-quark pair, called mesons,
or three-quark bound states, called baryons.
The anti-quark contained in a meson must have the corresponding anti-color of 
the constituent quark.
Similarly, the three quarks, which make up a baryon, must have colors that add
to ``white,'' meaning one red , one green, and one blue quark.

In particle collisions, this effect also leads to the phenomenon of ``jets,''
where a spray of hadrons is projected at the particle detector, resulting from
the multiple colored objects being pulled apart due to the energy of the
collision.

Since right-handed neutrinos are colorless objects which have have neither
isospin or electric charge, they are not expected to interact with the known
components of the SM, and are therefore not part of the theory.
There are extensions to the SM, however, which do include right-handed
neutrinos, including the one described in
Section~\ref{sec:theory_bl_extension}.

% The matter particles are described in the SM Lagrangian as quantum fields, with
% spin equal to \nicefrac{1}{2} (fermions).
% The full list of fermions, and the corresponding quantum numbers shown
% in Table~\ref{tab:sm_matter_content}.
% The matter content is broken up into three generations of particles, each with
% similar structure, differing by their mass.
% Each generation includes quarks, which have both electric and color, change,
% and therefore, interact with both the strong and electroweak forces.
% Leptons, including charged leptons and neutrinos, are also included in each
% generation, and have no color charge.
% Charged leptons, such as electrons have electric charge, while neutrinos do not.

% The weak force, mediated by the $W^{\pm}$ and $Z$ bosons, only interacts with
% matter fields with left-handed chirality, with non-zero isospin.
% In each generation, these left-handed fields make a quark doublet, comprising
% an up and down type quark, and a lepton doublet, containing a neutrino and a
% charged lepton.
% These doublets behave the same way under weak interactions, with the up-type
% quarks and neutrinos having the third component of weak isospin ($T_3$) equal 
% to $+\nicefrac{1}{2}$, and the down-type quarks and charged leptons having
% $T_3 = -\nicefrac{1}{2}$.
% The quarks and charged leptons each have a corresponding right-handed particle.
% These right-handed fermions are not part of an isospin doublet, and do not
% interact via the weak force.
% It should also be noted that, in the SM, neutrinos do not have a right-handed
% counterpart, as they would not interact via any known forces.

%% - - - - - - - - - - - - - - - - - - - - - - - - - - - - - - - - - - - - - - -
\FloatBarrier
\subsection{Higgs mechanism}
\label{sec:higgs}

While the photon is a massless particle, the $W^{\pm}$ and $Z$ bosons are
observed to be short ranged, implying they are massive.
The addition of these mass terms to the SM Lagrangian breaks gauge invariant,
and is, therefore, forbidden.
Similarly, the fermions have masses, which are forbidden by the SM.
Fermion masses would allow a mixing between the left- and right-handed sectors
fermions, which also breaks gauge invariance.

A method to add mass terms to the SM Lagrangian without breaking the necessary
gauge invariance was developed during the 1960's by several groups of theorists,
including Robert Brout and Francois Englert~\cite{PhysRevLett.13.321},
Peter Higgs~\cite{Higgs1964132,PhysRevLett.13.508},
and Gerald Guralnik, Carl R. Hagen, and Tom Kibble~\cite{PhysRevLett.13.585}
working in parallel, and is commonly called the Brout-Englert-Higgs (BEH)
mechanism.
The BEH mechanism adds an additional complex scaler field, known as the Higgs
field, to the Lagrangian, for which the ground state of the field is not at its
zero-point.
The Higgs field ``spontaneously'' breaks the gauge symmetry while preserving
the gauge invariance of the Lagrangian.

%% ------------------------------------------------------------------------------
\FloatBarrier
\section{Shortcomings}
\label{sec:sm_shortcomings}

%% -----------------------------------------------------------------------------
\FloatBarrier
\section{Supersymmetry}
\label{sec:susy}

Lightest supersymmetric particle (LSP) 

%% - - - - - - - - - - - - - - - - - - - - - - - - - - - - - - - - - - - - - - -
\subsection{R-Parity}
\label{sec:r_parity}

The extension of the Standard Model of particle physics with
supersymmetry (SUSY)~\cite{Miyazawa:1966,Ramond:1971gb,Golfand:1971iw,
Neveu:1971rx,Neveu:1971iv,Gervais:1971ji,Volkov:1973ix,Wess:1973kz,Wess:1974tw}
immediately leads to processes that violate both baryon number ($B$) and
lepton number ($L$), leading to rapid proton decay and
lepton-number-violating processes, such as unseen decays of
$\mu \to e\gamma$, in conflict with experimental bounds.
A conventional assumption to prevent these processes is to impose
conservation of $R$-parity~\cite{Fayet:1976et,Fayet:1977yc,Farrar:1978xj,
Fayet:1979sa,Dimopoulos:1981zb},
defined as $R=(-1)^{3(B-L)+2s}$ where $s$ is the spin of the particle.
This has a value of $+1$ for Standard Model particles and $-1$ for
SUSY particles.
In this case SUSY particles are produced in pairs, and the LSP is stable.
Further, this stable LSP cannot carry electric charge or color charge without
coming into conflict with astrophysical data.
At the LHC, the conventional experimental signature for SUSY particles
includes significant missing transverse momentum due to the non-interaction of
the LSP with the detector.

%% -----------------------------------------------------------------------------
\FloatBarrier
\section{B-L extension}
\label{sec:theory_bl_extension}


%% - - - - - - - - - - - - - - - - - - - - - - - - - - - - - - - - - - - - - - -
\FloatBarrier
\subsection{Motivation}

{\color{red} Why this model? :-) Talk about things like:
\begin{itemize}
\item RPC is overkill to prevent proton decay
\item Links neutrino sector to susy properties
\end{itemize}
}

An alternative approach is to add a local symmetry $U(1)_{B-L}$ to the
$SU(3)_C \times SU(2)_L \times U(1)_Y$ Standard Model with right-handed
neutrinos.
The minimal supersymmetric extension then only needs a vacuum expectation value
for a right-handed sneutrino in order to spontaneously break the
$B-L$ symmetry~\cite{FileviezPerez:2008sx, Barger:2008wn, FileviezPerez:2009gr,
Everett:2009vy, Evans:1986ada, Lukas:1998yy, Braun:2005ux, Braun:2005nv,
Braun:2006ae, Ambroso:2009jd, Ambroso:2010pe, Ovrut:2012wg}.
This minimal $B-L$ model violates lepton number but not baryon number, and is
consistent with proton stability and the bounds on lepton number violation.
The LSP can now decay via $R$-parity-violating (RPV) processes, and may now
carry color and electric charge.

%% - - - - - - - - - - - - - - - - - - - - - - - - - - - - - - - - - - - - - - -
\FloatBarrier
\subsection{Phenomenology}

This leads to unique signatures~\cite{FileviezPerez:2012mj, Perez:2013kla,
Ovrut:2012wg, Ovrut:2014rba, Ovrut:2015uea} that are disallowed in conventional
models with $R$-parity conservation.
The case where the LSP is a scalar top (stop) is most interesting
since, in general, the large mass of the top quark acts to make the
lightest stop significantly lighter than the other squarks due to
renormalization group effects~\cite{Barbieri:1987fn,deCarlos:1993yy}.
The stop decays via an RPV interaction to a charged lepton (of any
flavor) and a $b$-quark.
The decay branching fractions to $e b$, $\mu b$, and $\tau b$ may be different,
in a manner related to the neutrino mass
hierarchy seen in Figure~\ref{fig:pheno_bounds}.
Each point in this plot represents a simulation with a particular choice of
model parameters, all varied within a natural range of values, shown in
Table~\ref{tab:pheno_ranges}, and the four colors represent different choices
for the neutrino mass hierarchy and 
$\sin^2\theta_{23}$~\cite{Marshall:2014cwa,Marshall:2014kea}.
There is a clear relation between the neutrino mass hierarchy and the allowed
stop branching ratios, therefore if a stop consistent with this model is
discovered, its properties could potentially give information about the
structure of the neutrino sector.

\begin{figure}[p]
  \centering{
    \includegraphics[width=\textwidth]{figs/theory/WithGaussian.pdf}
    \caption{This plot shows the allowed branching ratios for the stop LSP
      for various choices of the neutrino mass hierarchy and
      $\sin^2\theta_{23}$,
      obtained by varying various model parameters within a range of natural
      values shown in Table~\ref{tab:pheno_ranges}~\cite{Marshall:2014cwa}.
    }
    \label{fig:pheno_bounds}
  }
\end{figure}

\begin{table}[ht]
  \caption{Ranges for the parameter scan used to generate the simulated models
    in \cref{fig:stop_vs_mixing_angle,fig:pheno_bounds}.
    The neutrino sector constrains all buy one one of the R-parity violating
    parameters, which is chosen to be $\epsilon_i$ where the generational
    index, $i$, is also scanned to avoid any biases.
    ``NH'' and ``IH'' represent the normal and inverted neutrino mass hierarchy
    respectively~\cite{Marshall:2014cwa}.
  }
  \label{tab:pheno_ranges}
  \centering{
    \begin{tabular}{cc}
      \toprule
      Parameter & Range \\
      \midrule
      $M_3~[\TeV]$               & $1.5 - 10$    \\[1ex]
      $M_{Z_\mathrm{R}}~[\TeV]$  & $2.5 - 10$    \\[1ex]
      $\tan\beta$                & $2 - 55$      \\[1ex]
      $\mu~[\GeV]$               & $150 - 1000$  \\[1ex]
      $m_{\stop_1}~[\GeV]$       & $400 - 1000$  \\[1ex]
      $\theta_t~[^\circ]$        & $0 - 90$      \\[1ex]
      $|\epsilon_i|~[\GeV]$      & $10^{-4} - 1$ \\[1ex]
      $\arg(\epsilon_i)$         & $0 - 360$     \\[1ex]
      $i$                        & $1 - 3$       \\[1ex]
      $\xi_0, \xi_3$             & $-1, 1$       \\[1ex]
      $\delta,\alpha~[^{\circ}]$ & $0 - 360$     \\[1ex]
      Neutrino Hierarchy         & NH, IH        \\
      \bottomrule
    \end{tabular}
  }
\end{table}

Within this model, a stop LSP can decay in one of two ways, depending on
the handedness of the stop.
A right-handed stop decays to a top quark and a right-handed neutrino with a
coupling strength proportional to vacuum expectation value (VEV) of the
left-handed neutrino mass.
This must be small because the left-handed sneutrino interacts with the
$W$ and $Z$ bosons, and a large VEV would result in these bosons gaining
additional mass.
{\color{red} (TODO check this statement is true - I think I'm missing
something).}
A purely left-handed stop decays to a $b$-quark and a lepton with the coupling
strength proportional to the VEV of the right-handed sneutrino, which may be
large, as the right-handed sneutrino does not couple to the electroweak bosons,
a large VEV does not break electroweak symmetry.
{\color{red} (TODO check this statement is true - I think I'm missing
something)}.
In the scenario where the stop LSP is an admixture of left- and right-handed
stops, the preferred decay mode depends on the stop mixing angle ($\theta_t$).
This dependence is plotted in Figure~\ref{fig:stop_br_vs_mixing_angle}, which
shows the ratio
$\nicefrac{\mathrm{Br}(\stop \to t\nu)}{\mathrm{Br}(\stop \to b\ell)}$
versus $\theta_t$.
Each point represents a simulation with a particular choice of model parameters,
again scanning over the natural values in Table~\ref{tab:pheno_ranges}.
The $\stop \to b\ell$ decay is the dominant decay mode for mixing angles less
than abut $80^{\circ}$, where the LSP stop is mostly right handed, and the
$\stop \to t\nu$ decay becomes significance.
The $\stop \to b\ell$ decay is still non-negligible for a mostly right-handed
stop in many of the simulated models,
however~\cite{Marshall:2014cwa,Marshall:2014kea}.

\begin{figure}[ht]
  \centering{
    \subbottom[Stop branching ratio]{
      % \includegraphics[width=0.70\textwidth]{figs/theory/StopBranchingRatiosVsMixingAngle.pdf}
      \includegraphics[width=\textwidth]{figs/theory/StopBranchingRatiosVsMixingAngle.pdf}
      \label{fig:stop_br_vs_mixing_angle}
    }
    \subbottom[Stop decay length]{
      % \includegraphics[width=0.70\textwidth]{figs/theory/DecayLength.pdf}
      \includegraphics[width=\textwidth]{figs/theory/DecayLength.pdf}
      \label{fig:stop_decay_length_vs_mixing_angle}
    }
    \caption{These plots show the stop decay length and branching ratio versus
      the stop mixing angle, assuming the stop is the LSP.
      Each point in these plots represent a simulation with a particular
      choice of model parameters, which are varies within a range of natural
      values~\cite{Marshall:2014cwa}.
    }
    \label{fig:stop_vs_mixing_angle}
  }
\end{figure}

The analysis described in this thesis focuses on the $\stop \to b\ell$ decay
as it is preferred for most of parameter space.
Additionally, if the $\stop \to t\nu$ decay is significant, the decay of stop
pairs would lead to final states with $t\bar{t}$ associated with large missing
energy, which is the same final state as stop pair production with
$R$-Parity conserving decays, and the limits from traditional stop searches can
be reinterpreted for this model.

It is also reasonable to assume the stop decays in this model are prompt, and
decay with a negligible impact parameter, as shown in
Figure~\ref{fig:stop_decay_length_vs_mixing_angle}, where for most natural
models, the stop is expected to have a decay length of less than
$10^{-3}~\mathrm{mm}$;
in particular, this is true for models where the stop is not mostly
right-handed ($\theta_t \leq 80$)~\cite{Marshall:2014cwa,Marshall:2014kea}.
Therefore, long-lived particles are not considered in this analysis.

In this $B-L$ extension to the MSSM, stop pair production has the same
production cross section as in the traditional MSSM.
The expected production cross section are shown in Figure~\ref{fig:stop_xsec}.

\begin{figure}[ht]
  \centering{
    \includegraphics[width=\textwidth]{figs/theory/xsec.pdf}
  }
  \caption{Stop cross sections and their associated
    uncertainties~\cite{Beenakker:1997ut,Beenakker:2010nq,Beenakker:2011fu}.
  }
  \label{fig:stop_xsec}
\end{figure}

% %% -----------------------------------------------------------------------------
% \FloatBarrier
% \subsection{Expected kinematics}
% 
% {\color{red} TODO talk about expected event kinematics at truth level}
% 
% The expected event kinematics are studied using Monte Carlo (MC) simulation.
% Simulated stop pair production events are generated using \madgraph\ version
% 1.5.12~\cite{Alwall:2011uj} and \pythia\ version 6.427~\cite{Sjostrand:2006za},
% and the expected event kinematics are studied at the truth level, ignoring
% detector effects and potential inefficiencies.
% The simulated signal event generation procedure is described in more detail in
% Section~\ref{sec:mc_samples}.
% For illustration purposes, three choices of stop mass (100~\GeV, 500~\GeV,
% and 1000~\GeV), are compared in this section to give a picture of how the event
% kinematics are expected to evolve with increasing stop mass.
% 
% Figure~\ref{fig:truth_stop_pt} shows the expected 

%% -----------------------------------------------------------------------------
\FloatBarrier
\subsection{Previous results}

The results from existing leptoquark searches performed at ATLAS were
re-interpreted in the context of this $B-L$ SUSY model with a stop LSP, and the
limits obtained on the minimum allowable stop mass across the plane of
physical stop branching ratios are shown in
Figure~\ref{fig:pheno_limit}~\cite{Marshall:2014cwa,Marshall:2014kea}.
The leptoquark searches, used to obtain these mass limits, searched for
models where the decay products are in the same generation.
As a result, $b$-tagging is only required in events associated with a $\tau$
lepton, and events with a light lepton simply require a jet, regardless of the
flavor.
Additionally, previous leptoquark searches only consider decays to a single
lepton flavor, resulting in final states with either two leptons of the same
flavor and two jets, or a single charged lepton and at least one jet.
The previous analyses did not consider final states with two charged leptons
with different flavors ($e\mu$) and at least two 
jets~\cite{ATLAS:2013oea, ATLAS:2012aq, Aad:2011ch, CMS:2014qpa,
  Chatrchyan:2012sv, Chatrchyan:2012vza, Chatrchyan:2012st}.
A dedicated search requiring $b$-tagged jets associated with light leptons
(electrons and muons), and considering different flavored leptons in the final
state can provide additional sensitivity to this model, as well as other models
which result in final states with $b$-quarks and light leptons in the final
state.

\begin{figure}[p]
  \centering{
    \includegraphics[width=\textwidth]
    {figs/theory/BoundContourPlotNoCombination.pdf}
    \caption{Limits on the stop mass obtained by reinterpreting leptoquark
      searches performed at ATLAS.
      The mass limits assume the stop is the LSP, and decays to a $b$-quark
      and a lepton~\cite{Marshall:2014cwa}.
    }
    \label{fig:pheno_limit}
  }
\end{figure}

