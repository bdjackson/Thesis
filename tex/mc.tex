%% \chapter[htoc-titlei][hhead-titlei]{htitlei}
%% -----------------------------------------------------------------------------
\chapter[Monte Carlo simulation][Monte Carlo simulation]{Monte Carlo simulation}
\label{ch:mc}

\begin{quote}
  Monte Carlo (MC) simulations are an important tool for particle physics
  experiments.
  MC techniques are used to simulate the physics processes that occur during
  particle collisions.
  These simulated events also include the interactions of the decay products
  in the detector, and can be used to tune the selection of an analysis,
  estimate the expected event yields and kinematic shapes, and ultimately,
  evaluate the expected sensitivity of a particular search.
  MC simulation of background and signal processes are used on \atlas.
  This chapter introduces some of the basic concepts of event generation, but
  focuses on the generation of the $B-L$ stop pairs from the model described in
  Section~\ref{sec:theory_bl_extension}.

  MC simulation of particle physics events can be broken into two major parts.
  The first step is the event generation, described in
  Section~\ref{sec:event_gen}.
  In the event generation stage, the actual Physics processes that occur as a
  result of the collision are simulated.
  This includes the hard and soft interaction as well as the resulting decay of
  any unstable particles, and finally the hadronization of any decay products
  with color charge.
  As in the real detector, once the proton collisions are simulated, the decay
  products travel through the detector, and may leave a measurable signature
  which can be measured.
  The simulation of these material interactions with the detector is discussed
  in Section~\ref{sec:det_sim}.
\end{quote}

%% -----------------------------------------------------------------------------
\FloatBarrier
\section{Event Generation}
\label{sec:event_gen}

\begin{figure}[p]
  \centering
  \includegraphics[width=\textwidth, clip=true, trim=0 0 0 0]
  {figs/mc_gen/full_mc_event.png}
  \caption[
    Pictorial representation of a $t\bar{t}H$ event as produced by an event
    generator~\cite{Gleisberg:2008ta}.
  ]{
    Pictorial representation of a $t\bar{t}H$ event as produced by an event
    generator.
    The hard interaction (big red blob) is followed by the decay of both top
    quarks and the Higgs boson (small red blobs).
    Additional hard QCD radiation is produced (red) and a secondary
    interaction takes place (purple blob) before the final-state partons
    hadronize (light green blobs) and hadrons decay (dark green blobs).
    Photon radiation occurs at any stage (yellow)~\cite{Gleisberg:2008ta}.
  }
  \label{fig:mc_event}
\end{figure}

Before discussing the details of event generation, it is useful to first
introduce the concept of an ``event.''
At the LHC, beams of protons are accelerated in opposite directions, and
allowed to cross at specific locations as described in Section~\ref{sec:lhc}.
At each of these crossings, protons from the two beams collide with one
another, resulting in a spray of particles in the detector.
Each of these crossings represents a single event.
Figure~\ref{fig:mc_event} shows the many levels of event generation in a
pictorial representation of a simulated $t\bar{t}H$ event.
Looking closely at the interactions taking place, one notices that,
rather than the full protons interacting with each other, the interactions
take place between the constituent quarks and gluons within the two protons.
In some cases, there is a ``hard interaction,'' or an inelastic scattering
where additional particles are created.
This is mathematically represented by the ``matrix element'' calculation.
The calculation of the hard interaction process is discussed in more detail
in Section~\ref{sec:matrix_element}.
Section~\ref{sec:parton_shower} introduces the ``parton shower'' technique,
used to model additional quarks and gluons radiated from the incoming and
outgoing partons in a collision event.
Combining the matrix element and parton shower techniques introduces a
potential double counting of certain event types.
The double counting is eliminated using a ``jet matching'' algorithm described
in Section~\ref{sec:jet_matching}.

There are also soft interactions between the remaining remnants of the proton
after the hard interaction in a process called the ``underlying event.''
The underlying event produces charged particles and jets in the detector, in
addition to those coming from the hard interaction~\cite{Field:2002vt}.

Other protons may interact in the same bunch crossing as well.
This is referred to as pileup.
As with the underlying event, pileup events produce additional charged
particles and jets that leave a signature in the detector.


%% - - - - - - - - - - - - - - - - - - - - - - - - - - - - - - - - - - - - - - -
% \FloatBarrier
\subsection{Parton distribution function}
\label{sec:pdf}

It is often said a proton is made up of three quarks and the gluons holding
them together.
Due to the QCD effects, this is actually a simplification, and additional
quarks and gluons are produced within hadrons, and have a large impact 
on the interactions between colliding protons.
The parton content within a hadron is described by a parton distribution
function (PDF).

The PDF of a proton, $p(x, \mu^2)$, gives the number of partons of a particular
flavor $p$ which are found in the proton with a fraction of the total momentum
between $x$ and $x+dx$, at a given energy scale $\mu^2$.
A proton must satisfy the following sum rules in order to have the correct
``valance quark'' content.
\begin{align}
  \int_0^1 \left[ u(x, \mu^2) - \bar{u}(x, \mu^2) \right] dx & = 2, \\
  \int_0^1 \left[ d(x, \mu^2) - \bar{d}(x, \mu^2) \right] dx & = 1, \\
  \int_0^1 \left[ q(x, \mu^2) - \bar{q}(x, \mu^2) \right] dx & = 0,
\end{align}
Where $q$ is any other quark flavor.
When the proton is described as $uud$, this refers to the valance quark
composition.

PDFs are tuned to the observed data at collider experiments such as \atlas,
and improve over time as more data is collected at higher energy scales.
Two example PDFs, provided by the NNPDF group, are shown in
Figure~\ref{fig:pdfs}, corresponding to different energy scales.
The CTEQ 6L1 PDF tune is used for the simulated stop pair production in the
analysis described in this thesis~\cite{Nadolsky:2008zw}.

\begin{figure}
  \centering
  \includegraphics[width=0.80\textwidth]{figs/mc_gen/pdfs.png}
  \caption{Parton distribution function (PDF) distributions for different
    parton species within the proton as obtained by the NNPDF collaboration.
    The two PDF distributions are obtained for different energy
    scales ($\mu^2$)~\cite{nnpdf}.
  }
  \label{fig:pdfs}
\end{figure}


%% %% - - - - - - - - - - - - - - - - - - - - - - - - - - - - - - - - - - - - - - -
%% \FloatBarrier
%% \subsection{Underlying event}
%% \label{sec:underlying_event}
%% 
%% The underlying event makes up the parts of the collision event resulting from
%% the breakup of the initial-state protons after partons are knocked out during
%% the hard interaction.
%% A schematic of a di-jet
%% the underlying event highlighted in a di-jet event is shown in
%% Figure~\ref{fig:underlying_event}.
%% The different lines represent the incoming protons (blue), the hard
%% interaction (red), the underlying event (black), and initial and final state
%% radiation (pink).
%% The hard interaction includes the incoming partons which interact, the
%% resulting outgoing particles which go on to form the final-state jets, and any
%% final state or initial state radiation emitted from those particles.
%% 
%% \begin{figure}
%%   \centering
%%   \includegraphics[width=0.80\textwidth]{figs/mc_gen/underlying_event.png}
%%   \caption[
%%     Schematic of a di-jet event with the underlying event in a
%%     proton-anti-proton collision~\cite{Field:2002vt}.
%%   ]{
%%     Schematic of a di-jet event with the underlying event in a
%%     proton-anti-proton collision.
%%     This diagram looks similar for proton-proton
%%     collisions~\cite{Field:2002vt}.
%%   }
%%   \label{fig:underlying_event}
%% \end{figure}
%% 
%% The proton breaks apart during the collision, and the ``beam remnants'' are
%% scattered, and potentially undergo additional, softer interactions.
%% This is known as ``multiple parton interactions,'' and leads to additional
%% activity in the detector, including additional charged particles and jets.
%% There can also be initial or final state radiation associated with the
%% beam remnant, which is also classified as part of the underlying
%% event~\cite{Field:2002vt}.
%% 
%% Modeling the underlying event requires adequate tunes of the proton PDFs
%% and the soft interaction terms, and is beyond the scope of this thesis.

%% - - - - - - - - - - - - - - - - - - - - - - - - - - - - - - - - - - - - - - -
\FloatBarrier
\subsection{Hard interaction}
\label{sec:matrix_element}

The ``hard interaction'' refers to the process within a collision event
with the highest $\sum \pt$ when summing over all the outgoing particles
from a particular interaction.
This is usually the most interesting part of the event, so MC simulation of
collision events are generally categorized by the hard interaction process.

The cross section of a generic scattering process $a\,b \to F+X$ is given by
\begin{equation}
  \sigma_{a\,b \to F+X} \propto
  \left|\mathcal{M}_{a\,b \to F+X}\right|^2 \Phi_{F},
\end{equation}
where $a$ and $b$ are the incoming partons, $F$ is the final state, and $X$
represents anything else that may be in the event, such as additional quarks or
gluons which go on to form jets.
The production cross section is given in terms of a matrix
element, $\mathcal{M}_{a\,b \to F+X}$, and a phase space term, $\Phi_F$.
The matrix element term is $\mathcal{M}_{a\,b \to F+X}$ is the scattering
amplitude, and is given by the scattering potential.
% amplitude, and is given by the Fourier Transform of the scattering potential.
The phase space is related to the density of final states.

Several software tools exist to calculate the scattering cross section by
generating the relevant diagrams, calculating these matrix element terms,
and integrating over the phase space for the final states.
The tools then use these cross sections, and Monte Carlo techniques, to simulate
collision events, which have the same production cross section and kinematics
as events one expects to observe in real collision data.
The simulated events mimic the realistic collision events in that they have
incoming and outgoing particles, and any unstable particles are allowed to
decay.
Each of the particles in the simulated event are represented by four-vectors,
with values for the mass and momentum.
Examples of these software tools include \sherpa, \powheg, and \madgraph.
In particular, \madgraph\ is used to generate the simulated signal events for
the analysis described in this thesis.

%% - - - - - - - - - - - - - - - - - - - - - - - - - - - - - - - - - - - - - - -
\FloatBarrier
\subsection{Parton shower}
\label{sec:parton_shower}

The matrix element calculations are a useful tool in calculating the cross
sections for processes with high \pt\ and well separated particles in the
final state,
% infrared safe processes such as $Z$~boson or di-jet production,
but there can be cases, such as soft or collinear radiation, where the
calculations diverge.
These divergences are clearly non-physical, as such infinities don't exist in
nature, so a different approach is needed in this regime.
A common technique is the ``parton shower'' method implemented by software
tools such as \pythia.
The basic idea of the parton shower method can be illustrated in the context of
gluon radiation off of a quark.
Similar to Bremsstrahlung radiation of QED, a quark will radiate gluons as it
travels.
\pythia\ models the quark as moving along its trajectory, and in each step,
there is some probability of radiating a gluon with a given momentum.
Each time a gluon is radiated, the quark will lose momentum, until a
threshold is reached, and the quark will begin to hadronize rather than
continue to radiate.

%% - - - - - - - - - - - - - - - - - - - - - - - - - - - - - - - - - - - - - - -
\FloatBarrier
\subsection{Jet matching}
\label{sec:jet_matching}

As with the matrix element calculation, the parton shower method is not without
its problems.
The parton shower is not as well suited for calculating the cross section
for very hard processes, where the momentum transfer is large, but this is
exactly the regime where the matrix element calculations is accurate.
For this reason, both methods are used when simulating collision events for
use in analyses.
The matrix element is used to generate the hard process and the initial decay
products, then the parton shower takes over to add any additional radiation
below the cutoff scale of the matrix element calculation.

Naively combining the matrix element calculation with the parton shower
technique introduces a potential double counting of diagrams as illustrated in
Figure~\ref{fig:mc_double_counting}.
Figure~\ref{fig:mc_double_counting} shows three diagrams for the production
of a $Z$~boson associated with additional partons, which lead to jets in the
final state.
The matrix element technique is used to generate the $Z$~boson
production along with one or two additional partons.
Parton showering is applied to the additional partons, but not the initial
state particles to simplify the example, but in a complete simulation, parton
showering is applied to all partons.

In the left scenario only a single additional parton is included in the matrix
element calculation, and the parton shower method is used to determine the soft
collinear radiation off of the quark.
One would like to add scenarios like the middle diagram, which includes a
second additional parton in the matrix element calculation.
The middle diagram includes a hard emission of a gluon in the matrix element
calculation, and parton showering is used to include the soft radiation from
the gluon.
If these are the only two diagrams, there would be no problem, however, the
parton shower will occasionally result in a hard, large-angle gluon emission
from the quark, as shown in the right scenario.
This leads to the possibility of double counting since both the middle and
right diagrams generate the same process~\cite{Salam:2010zt}.

\begin{figure}
  \centering
  \includegraphics[width=0.8\textwidth]{figs/mc_gen/double_conting.png}
  \caption[
    Illustration of the double counting issue that arises when combining the
    matrix element and parton shower event generation
    techniques~\cite{Salam:2010zt}.
  ]{
    Illustration of the double counting issue that arises when combining the
    matrix element and parton shower event generation techniques.
    The diagrams show $Z$~boson production associated with one or two
    additional partons included in the matrix element
    calculation~\cite{Salam:2010zt}.
  }
  \label{fig:mc_double_counting}
\end{figure}

As mentioned before, the matrix element technique is well suited for processes
with large momentum transfer, while parton showering is better for soft,
collinear emissions.
It is necessary to define a cutoff for when an additional parton emission is
``hard enough'' to be included in the matrix element calculation.
This procedure of eliminating double counting is called ``jet matching,'' and
many methods exist.
While the implementation differs for each of the algorithms, the basic approach
for each of the methods is illustrated in Figure~\ref{fig:mc_jet_matching}.
A scale is defined for each of the additional partons produced, either in the
matrix element calculation, or through the parton showering.
The definition of the scale is different for each algorithm, but is it related
to the momentum of the parton and the angle at which it is radiated.
In this discussion, the scale will be referred to as $k_\mathrm{T}$.

Two thresholds are defined, called $Q_\mathrm{ME}$ and $Q_\mathrm{merge}$,
where $Q_\mathrm{ME} \leq Q_\mathrm{merge}$.
These thresholds mark the cutoff between the matrix element and parton
showering stages of the event generation process.
Partons generated in the matrix element calculation are required to have
$k_\mathrm{T} \geq Q_\mathrm{ME}$.
Similarly, the parton shower step may only add additional partons with
$k_\mathrm{T} \leq Q_\mathrm{ME}$~\cite{Salam:2010zt}.

For the analysis described in this thesis, the Shower $k_\mathrm{T}$ is used
when generating stop pair events, with thresholds chosen such that
% $Q_\mathrm{ME} = Q_\mathrm{merge} = \frac{m_{\tilde{t}}}{4}$~\cite{Alwall:2008qv}.
$Q_\mathrm{ME} = Q_\mathrm{merge} = \frac{1}{4}m_{\tilde{t}}$~\cite{Alwall:2008qv}.

\begin{figure}
  \centering
  \includegraphics[width=0.8\textwidth]{figs/mc_gen/jet_matching.png}
  \caption[
    Illustration of the jet matching technique used to eliminate the double
    counting issue shown in
    Figure~\ref{fig:mc_double_counting}~\cite{Salam:2010zt}.
  ]{
    Illustration of the jet matching technique used to eliminate the double
    counting issue shown in
    Figure~\ref{fig:mc_double_counting}.
    The $Q_\mathrm{merge}$ cutoff scale is shown as a dashed blue line, and
    events failing this requirement are rejected~\cite{Salam:2010zt}.
  }
  \label{fig:mc_jet_matching}
\end{figure}


% {\color{red} Talk about matching. Probably use several figures to help explain
%   this topic. Probably also talk about these DJR plots, and how we want them
%   to look}
% 
% {\color{red} Add info about HFOR here}

%% -----------------------------------------------------------------------------
\FloatBarrier
\section{Detector simulation}
\label{sec:det_sim}

The response of the detector in a candidate $t\bar{t}H$ event is shown in
Figure~\ref{fig:data_event}.
It is the task of the detector simulation step to transform the list of
particles from Figure~\ref{fig:mc_event} into the energy deposits shown in the
detector in Figure~\ref{fig:data_event}.
The reconstruction described in Section~\ref{sec:event_reco} interprets these
signals in exactly the same way for simulation and data.
The interactions of particles with the material in the detector are
simulated in one of two ways, know as full simulation or fast simulation.

\begin{figure}[ht]
  \centering
  \includegraphics[width=\textwidth, clip=true, trim=0 0 0 0]
  {figs/mc_gen/tth_event.png}
  \caption{
    Candidate $t\bar{t}H$ event, observed in the
    \atlas\ detector~\cite{Aad:2015iha}.
  }
  \label{fig:data_event}
\end{figure}

In the full simulation, a model of the \atlas\ detector is built within \geant4.
The simulation includes, not only, the active parts of the detector used in
measuring particles, but also the support structure, and services such as
cables and electronics.
The particles are propagated through the simulated detector, and allowed to
interact in a way similar to a real particle traveling through the
\atlas\ detector.
Simulated interaction in any of the subsystems result in simulated
measurements, which can be put into the reconstruction software, just as
data from the real detector.

Much like the full simulation, the fast simulation uses \geant4 to simulate the
interactions of particles through the inner detector and the muon systems.
The two models differ in the treatment of the calorimeter.
Rather than fully simulate the showers in the calorimetry system, the fast
simulation uses parameterized shower shapes for particles that interact with
the electromagnetic and hadronic calorimeters.

A more complete discussion of the \atlas\ simulation setup can be found
in Reference~\cite{ATLASSimulation}.

