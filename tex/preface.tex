
\chapter*{Preface}
\addcontentsline{toc}{chapter}{Preface}

My time as a graduate student as the University of Pennsylvania has been an
amazing experience, where I learned an incredible amount, and met many amazing
people.
I was around for the startup of the Large Hadron Collider (LHC) and the
\atlas\ experiment, and was able to take part in the first LHC run.
I am very fortunate to have had the opportunity to work on the LHC during this
exciting time.
In particular, being part of the \atlas\ group at the University of Pennsylvania
has been a wonderful experience, offering me the opportunity to take on
many responsibilities and grow as a physicist.

I began my graduate student career in 2008, when I spent my first summer at
CERN learning about \atlas, the TRT, and the software tools that make
everything work together.
I even had the opportunity to get my hands dirty, repairing TRT patch panels,
and climbing through the \atlas\ detector to reinstall them.
This was immediately followed by me moving to Philadelphia to take classes.
I still found time to continue my work on TRT detector performance, including
several studies of the TRT timing resolution, documented in
ATL-COM-INDET-2010-045 and
ATL-COM-INDET-2011-007~\cite{Cornelissen:1255892,Fratina:1340946}.

Once my classes were complete in 2010, I moved to CERN in order to immerse
myself in research.
I continued my work on the TRT, and began studying muon isolation in the muon
combined performance group.
I also joined the search for a massive $Z'$~boson, where I performed several
small studies.
The $Z'$~search was my first introduction to the world of physics analysis.

In 2012, I switched from exotics to the Supersymmetry group to continue my
research.
During 2012 and 2013, I worked on a team, searching for SUSY signatures,
produced through electroweak interactions.
These include direct production of electroweak gauginos and sleptons.
I performed many cross check studies, worked on signal region optimization,
and wrote and maintained tools to provide a ``fake lepton'' background
estimate for the rest of the analysis group.
Additionally, I wrote an analysis framework used by several colleagues at Penn
and Harvard to analyze the large \atlas\ datasets and Monte Carlo simulation
samples.
These two years taught me a lot about software development, analysis
techniques, and working on a large team toward a common goal.

Finally, toward the end of 2013, I was ready to take on my own analysis, which
became the topic of this thesis.
Through conversations with Burt Ovrut, Sogee Spinner, and Austin Purves, I
became interested in SUSY models which allow for $R$-Parity violation.
These models provide a large range of experimental signatures, not covered by
other searches at \atlas.
I chose to pursue a completely new search, looking for a stop LSP, which
decayed as a leptoquark.
While there are many leptoquark searches, a direct search for this model 
provides additional sensitivity compared to reinterpretations of other searches.
Ultimately, we extended the limits on the allowable stop masses, as described
in this thesis and in the conference note
ATLAS-CONF-2015-015~\cite{ATLAS-CONF-2015-015}.

In addition to myself, the analysis team consisted Leigh Schaefer, Evelyn
Thomson, and Joseph Kroll and Zachary Marshall who acted as additional advisers.
Being the primary analyzer on such a small team gave me the opportunity to take
part in every piece of the analysis, from the Monte Carlo simulation of our
simulated signal samples to the selection optimization, and finally the
background estimates and limit setting.
This gave me a strong feeling of ownership of the work.
Additionally, working with Leigh allowed me to experience what it's like to
train a junior graduate student, which was one of the most enjoyable things
I did during my time as a graduate student.
All this this work culminated in a conference note that was presented at the
Moriond~2015 conference.

The \lhc\ and the \atlas\ experiment are extremely complex machines, so I
reference other documents heavily while describing them in this this thesis.
Additionally, I took much of the text from the conference
note~{ATLAS-CONF-2015-015}, and the corresponding internal support
note~{ATL-COM-PHYS-2015-168} verbatim, as I was the primary author.
When sourcing material, for which I am not the primary author, I attempted to
rewrite the text in my own words.

I would like to wish the \atlas\ experiment and the Penn army good luck in
Run~II!
It will surely be an exciting time.

\vspace{0.05\textheight}

\begin{tabular}{p{0.5\textwidth} l}
  & Brett Jackson         \\
  & Philadelphia, July 2015   \\
\end{tabular}

